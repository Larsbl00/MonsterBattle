\subsection*{Launching}

When launching the program you can the {\ttfamily -\/h} or {\ttfamily -\/-\/help} flags to see all launch parameters. When launching the game, make sure you add the {\ttfamily -\/t} flag to select a trainer file for the player, and use either {\ttfamily -\/c} or {\ttfamily -\/o} to select an opponnent.

The following table will explain all parameters\+:

\tabulinesep=1mm
\begin{longtabu} spread 0pt [c]{*{3}{|X[-1]}|}
\hline
\rowcolor{\tableheadbgcolor}\textbf{ Short }&\textbf{ Long }&\textbf{ Description  }\\\cline{1-3}
\endfirsthead
\hline
\endfoot
\hline
\rowcolor{\tableheadbgcolor}\textbf{ Short }&\textbf{ Long }&\textbf{ Description  }\\\cline{1-3}
\endhead
-\/h &--help &Shows all flags and waht they do \\\cline{1-3}
-\/t &--trainer &Selects a trainerfile that the main player uses \\\cline{1-3}
-\/o &--opponent &Selects a trainer file for the oppononent to use, said opponent should be controlled by another person \\\cline{1-3}
-\/c &--cpu-\/trainer &Selects a trainer file for the opponent which is controlled by the game itself \\\cline{1-3}
\end{longtabu}
\subsection*{Assets}

The {\ttfamily assets} directory contains all files for the game, in all of these files you can use {\ttfamily \#} to create lines of comments. All default files have an explanation on how you should format everything. This part of the Read\+Me will also tell you how to alter the information.

\subsubsection*{Trainers}

A trainer has a collection of different monsters used to attack the opponent, which is also a trainer. When launching the program you can use different parameters to create a new trainer, one of the parameters is a \textquotesingle{}Trainer-\/file\textquotesingle{} the file has a very simple construct.

Per line in said file you need to use the following format\+: 
\begin{DoxyCode}
\{NAME, NICK\_NAME\}

*NAME: The name of the monster
*NICK\_NAME: The name you want to display for your monster
\end{DoxyCode}


You can select a total of {\ttfamily 6} monsters, you can also decide to choose fewer.

\subsubsection*{Moves}

You can easily add new moves to the game, we\textquotesingle{}ll go over how to use those moves later. To add a new move open {\ttfamily assets/moves.\+txt} and add a new line for your move; said move should be formatted as follows\+: 
\begin{DoxyCode}
(NAME,TYPE\_AS\_INTEGER,DAMAGE,PRECISION)


*NAME:  The name of the move, e.g. 'Yeet'

*TYPE\_AS\_INTEGER: The typing of the move, i.e. if I want it to be a fire move I'll open `include/Type.h`
       and check what number fire is, in this case it is 1

*DAMAGE: The damage of the move, with ~50 being equal to normal damage

*PRECISION: A NORMALIZED probability of the move hitting, e.g 0.5 (which is a mediocre accuracy)
\end{DoxyCode}
 Do remeber the name you assign to a move, as you\textquotesingle{}ll need this later when linking the move.

\subsubsection*{Models}

You can create a custom Model for a monster by creating a file containing an A\+S\+C\+II model for a monster and then linking it in {\ttfamily assets/monsters.\+txt}

\subsubsection*{Monsters}

Like attack moves, you can easily add more Monsters to the game. To add a new monster open the file {\ttfamily assets/monsters.\+txt} and add a line for your new monster. However do make sure that you add a new monster in the format used in the file itself, the format is as follows\+: 
\begin{DoxyCode}
(NICK\_NAME; NAME; TYPES; STATS; MOVES; MODEL\_FILE)


*NICK\_NAME: The nickname of the Monster, normally it is equal to the Monster's name

*NAME: The name of the monster, e.g. SmallBird

*Types: The types of the monster, like with the moves the types should be written down as integers. The
       Types should be written as follows [TYPE1,TYPE2], if I'd want a Fire type monster it is written as [1,0]

*Stats: The stats of the monster, the stats are formatted as follows:
  \{MAX\_HP, HP, DEF, ATTACK, PRECISION, AVOIDANCE, SPEED\}
  *note: PRECISION and AVOIDANCE should be normalized values

*Moves: This is a list of all moves that the monsater has, to add moves use the following format:
  [MV1,MV2,MV3,MV4]
  Each monster can have a maximum of 4 moves, if you want to use you can write the name down on any of the
       available slots, do make sure that they are present in "assets/moves.txt"

*ModelFile: The path to the file containing an ASCII-model, refer to the SmallBird example in the monster
       file
\end{DoxyCode}
 